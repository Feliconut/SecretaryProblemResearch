We give the expected value of $\ALG$ on the relaxed model. Let $\alpha$ denote the selected item.

\begin{equation}
  \expect{\OPT}=m
\end{equation}

\begin{equation}
  \expect{\ALG}=\sum_{t=1}^m t\prob{v(\alpha)=t}
\end{equation}

\begin{align}
  \prob{v(\alpha)=t} &= \sum_{i=r}^n
  \prob{\alpha = i | v(i)=t}
  \prob{v(i)=t}\\&+
  \prob{\text{reach }n \cap \text{not accept }n} \prob{v(n)=t}\\&+
  \prob{\text{reach }n-1 \cap \text{examine and decline }n-1}\prob{v(n-1)=t}
\end{align}

The two added terms are cases where the items are exhausted. If $n$ is reached then it must be selected. If $n-1$ is examined then it must be selected also.

\begin{equation}
  \prob{v(i)=t}=1/m, \forall i\forall t
\end{equation}

\begin{align}
  \prob{\alpha = i | v(i)=t} &= \prob{\text{reach }i|\max_{j<r}v(j)\leq t}\prob{\max_{j<r}v(j)\leq t}\\&=
  \sum_{s=1}^t \prob{\text{reach }i|\max_{j<r}v(j)= s}\prob{\max_{j<r}v(j)= s}
\end{align}

\begin{equation}\label{eqn:exhaust1}
	\prob{\text{reach }n \cap \text{not accept }n}
	=\sum_{s=t+1}^m
		\prob{\text{reach }n|\max_{j<r}v(j)= s}\prob{\max_{j<r}v(j)= s}
\end{equation}

\begin{equation}\label{eqn:exhaust2}
	\prob{\text{reach }n-1 \cap \text{examine and decline }n-1}
	=\sum_{s=t+1}^m
		\ind_{s\leq t + \delta}
		\prob{\text{reach }n-1|\max_{j<r}v(j)= s}\prob{\max_{j<r}v(j)= s}
\end{equation}

\begin{align}
\prob{\max_{j<r}v(j)= s} 
	&= \prob{\max_{j<r}v(j)\leq s}-\prob{\max_{j<r}v(j)\leq s-1}\\
	&= (\frac{s}{m})^{r-1}-(\frac{s-1}{m})^{r-1}
\end{align}

\begin{align}
  \prob{\text{reach }i|\max_{j<r}v(j)= s}
  &= \sum_{(a_1,\dots,a_k):a_j\in\{1,2\},\sum a_j = i-r} \,\prod_{j=1}^k 
  \ind_{\delta<t}
  \left(\ind_{a_j = 1}\frac{s-\delta-1}{m}
  +\ind_{a_j = 2}\frac{\delta}{m}
  \right)
  +\ind_{\delta\geq t}\frac{t-1}{m}
  \\
  &= \sum_{x=0}^{\lfloor\frac{i-r}{2}\rfloor} 
  \left(
  		\ind_{\delta<t}\left(\frac{\delta}{m}\right)^x\left(\frac{s-\delta-1}{m}\right)^{i-r-2x}
  		+\ind_{\delta \geq t} \left(\frac{t-1}{m}\right)^x 0^{i-r-2x}
  \right)
  \frac{(i-r-x)!}{x!(i-r-2x)!}
\end{align}

Explanation: There are $i-r$ steps to go from $r$ to $i$. Each decline will go 1 step, and each examine+decline will go 2 steps. We calculate the probability to take each possible path and then sum them up. In the second equation, $x$ is the number of items examined going from $r$ to $i$.
