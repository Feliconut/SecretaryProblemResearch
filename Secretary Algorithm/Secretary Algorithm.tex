\documentclass{article}
\usepackage{algorithm}
\usepackage{algpseudocode}
\usepackage{pifont}
\usepackage{amsmath}
\usepackage{amsfonts}
\usepackage{mathtools}

\newcommand{\oP}{\operatorname{P}}

\newcommand{\expec}{\operatorname{E}}
\newcommand{\gt}{>}
\newcommand{\lt}{<}
\newcommand{\card}[1]{\lvert #1 \rvert}

\newcommand{\expect}{\operatorname{E}\expectarg}
\newcommand{\prob}{\operatorname{P}\expectarg}
\DeclarePairedDelimiterX{\expectarg}[1]{[}{]}{%
  \ifnum\currentgrouptype=16 \else\begingroup\fi
  \activatebar#1
  \ifnum\currentgrouptype=16 \else\endgroup\fi
}

\newcommand{\innermid}{\nonscript\;\delimsize\vert\nonscript\;}
\newcommand{\activatebar}{%
  \begingroup\lccode`\~=`\|
  \lowercase{\endgroup\let~}\innermid 
  \mathcode`|=\string"8000
}

\begin{document}

%$i \approx j$ means that $i$ and $j$ are in-distinguishable in the first observation.
%    
%    
%    \begin{itemize}
%  \item (reflexivity) $i \approx i$
%   \item(symmetry) $i \approx j \iff j \approx i$
%   \item(connectedness) $i \approx j , r_i < r_j \implies \forall k, l [r_k, r_l \in [r_i, r_j] \implies r_k \approx r_l ]$
%\end{itemize}
%




\begin{algorithm}
\caption{Oct 14 2021}
\label{algo1} 
\begin{algorithmic}[1]

\For{$i \in [n]$}
	\State{$\alpha \gets$ the current item to be decided}
	\If{$i < \lfloor \frac{n}{e}\rfloor$} \Comment Phase 1
		\If{$\exists \beta\in O_i \ldotp \alpha <_i \beta$}\Comment $\alpha$ is \emph{not} a maximum
			\State{Decline this item}
		\ElsIf{$\alpha \in E_i$}
			\State{Decline this item}
		\Else 
			\State{Examine this item}
		\EndIf
			
	\ElsIf{$i < n$}\Comment Phase 2
		\If{$\exists \beta\in O_i \ldotp \alpha <_i \beta$}
			\State{Decline this item}
		\Else				\If{$\forall \beta\in O_i \ldotp \beta \leq_i \alpha$} \Comment $\alpha$ is the absolute maximum
				\State{Accept this item}
			\ElsIf{$\alpha \in E_i$}
				\State{Decline this item}
			\Else
				\State{Examine this item}
			\EndIf
		\EndIf
	\Else
		\State{Accept this item}
	\EndIf

\EndFor

\end{algorithmic}
\end{algorithm}




\begin{algorithm}
\caption{Game Process}
\label{algo1} 
\begin{algorithmic}[1]


\State{$[n], r_i$ and $\approx$ relation are given according to definition}


\State{$O_1 = \{1\}$}
\State{$E_1 = \emptyset$}
\State{$\alpha_1 = 1$}
\State{$\Theta_1 = \emptyset$}



\For{$i \in [n]$}
	\State{$d_{i} \gets $ player's decision}
	\If{$d_i = \mathrm{Accept}$}
		\If{$\alpha_i = \mathrm{argmax}_j r_j$}
			\State Game terminates and player wins
		\Else
			\State Game terminates and player loses
			\EndIf
			
	\ElsIf{$d_i = \mathrm{Decline}$}
		\State{$O_{i+1} \gets O_i\cup \{i+1\}$}
		\State{$E_{i+1} \gets E_i$}
		\State{$\alpha_{i+1} \gets i+1$}
		
	\ElsIf{$d_i = \mathrm{Examine}$}
		\State{$O_{i+1} \gets O_i$}
		\State{$E_{i+1} \gets E_i\cup \{i\}$}
		\State{$\alpha_{i+1} \gets \alpha$}
	\EndIf
	\State{$\Theta_{i+1} \gets 
		\{(\beta, \gamma) \in O_{i+1}^2: r_\beta < r_\gamma, \beta \not\approx \gamma\} \cup
		\{(\beta, \gamma) \in E_{i+1}^2: r_\beta < r_\gamma\}$}
	
	
\EndFor

\end{algorithmic}
\end{algorithm}

\pagebreak

\newcommand{\ALG}{\text{ALG}}
\newcommand{\BestSoFar}{{\pi_{i-1}}}

We first observe that $O_i$ is a uniformly random sample from $[n]$. Therefore, the apparent rank of any upcoming item $\alpha_i$ is uniformly distributed in $[s_i]$, where $s_i = |O_i|$. We denote the best item observed so far by $\BestSoFar = \sup O_{i-1}$.

In particular, $\alpha_i >_r \BestSoFar$ with probability $1/s_i$. In phase 2, this is the necessary and sufficient condition for $\alpha_i$ to be accepted (A) by $\ALG$.

From now on we assume that $\alpha_i = i$. Define $N_j := \{k\in[n] | k \approx j\}$ be the neighbour of item $j$. Under the assumption that $\alpha \approx \beta \iff |r_\alpha - r_\beta|\leq \epsilon$, we have $|N_i|=2 \epsilon + 1$ whenever $r_i \in [\epsilon+1, n-\epsilon]$.

We put $i$ into one of four types. 
\begin{itemize}
\item It is a type-I item if $\alpha_i >_r \BestSoFar$ and $i \not \approx \BestSoFar$. A type-I item is examined and declined (ED) in phase I, and accepted (A) directly in phase II.

\item $i$ is a type-II item if $\alpha_i >_r \BestSoFar$ and $i \approx \BestSoFar$, which means that it's still the best ever and close enough to the previous best. A type-II item is ED in phase I, and examined and accepted (EA) in phase II. 

\item $i$ is a type-III item if $\alpha_i <_r \BestSoFar$ and $i \approx \BestSoFar$, which means that it's worse than the previous best but close enough to it. A type-III item is ED in both phase I and phase II. 

\item $i$ is a type-IV item if $\alpha_i <_r \BestSoFar$ and $i \not \approx \BestSoFar$, which means that it's much worse than the previous best and not even close to it. A type-IV item is declined directly (D) in both phase I and phase II. 
\end{itemize}

We already know $\prob{i \text{ is type-I or type-II}} = 1/s_i$. In case $r_\BestSoFar < \epsilon+1$, $i$ cannot be type-I. In case $r_\BestSoFar > n-\epsilon$, $i$ cannot be type-IV.

Conditioned on $r_\BestSoFar > \epsilon+1$, $\prob{i \text{ is type-II}}= \frac{\epsilon}{n-1}$, because $i$ is a random element picked from the remaining $(n-1)$ items, and it's type-II iff it falls into half a neighborhood of size $\epsilon$. For the same reason, $\prob{i \text{ is type-III} | r_\BestSoFar < n - \epsilon}= \frac{\epsilon}{n-1}$.

Combining all we know, we work out the probability for $i$ to fall in each type. We denote $p_1 = \prob{i \text{ is type-I}}$, $p_2 = \prob{i \text{ is type-II}}$, etc.

\begin{align}
p_1 &= \max(1/s_i - p, 0)\\
p_2 &= \min(p, 1/s_i)\\
p_3 &= \min(p, 1-1/s_i)\\
p_4 &= \max(1-1/s_i-p, 0)
\end{align}

where $p = \frac{\epsilon}{n-1}$.

%In case of D, $O_{i+1} = O_i + \{\alpha_i\}$ and the process continues for next $i$. ED forgoes one item un-observed. Hence in case of ED, $O_{i+2} = O_i + \{\alpha_i\}$ and the process continues for $i+2$.

We now develop a Markov Chain model. The game state can be expressed by $(i,s)$ with initial state $(1,1)$. The state transition probabilities are:

\begin{align}
\prob{(i,s)\to (i+1, s+1)} &= \begin{cases}
	p_4 & i < n\\
	0 & i = n
	\end{cases}\\
\prob{(i,s)\to (i+2, s+1)} &= 
	\begin{cases}
	p_1 + p_2 + p_3 & i < r\\
	p_2 + p_3 & r \leq i \leq n-2\\
	0 & n-1 \leq i \leq n
	\end{cases}\\
\prob{(i,s)\to \infty} &= 
	\begin{cases}
	0 & i < r\\
	p_1 & r \leq i \leq n-2 \\
	p_1 + p_2 + p_3 & i = n-1\\
	1 & i = n
	\end{cases}
\end{align}


The winning rate of $\ALG$ is

\begin{equation}
  \prob{\text{Win}}=\frac{1}{n}\sum_{i=r}^n \sum_{s\in [i/2,i]} \prob{(1,1) \to (i,s)}
\end{equation}



\end{document} 