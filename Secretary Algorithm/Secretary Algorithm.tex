\documentclass{article}
\usepackage{algorithm}
\usepackage{algpseudocode}
\usepackage{pifont}
\usepackage{amsmath}
\usepackage{amsfonts}
\usepackage{mathtools}

\newcommand{\oP}{\operatorname{P}}

\newcommand{\expec}{\operatorname{E}}
\newcommand{\gt}{>}
\newcommand{\lt}{<}
\newcommand{\card}[1]{\lvert #1 \rvert}

\newcommand{\expect}{\operatorname{E}\expectarg}
\newcommand{\prob}{\operatorname{P}\expectarg}
\DeclarePairedDelimiterX{\expectarg}[1]{[}{]}{%
  \ifnum\currentgrouptype=16 \else\begingroup\fi
  \activatebar#1
  \ifnum\currentgrouptype=16 \else\endgroup\fi
}

\newcommand{\innermid}{\nonscript\;\delimsize\vert\nonscript\;}
\newcommand{\activatebar}{%
  \begingroup\lccode`\~=`\|
  \lowercase{\endgroup\let~}\innermid 
  \mathcode`|=\string"8000
}

\begin{document}

%$i \approx j$ means that $i$ and $j$ are in-distinguishable in the first observation.
%    
%    
%    \begin{itemize}
%  \item (reflexivity) $i \approx i$
%   \item(symmetry) $i \approx j \iff j \approx i$
%   \item(connectedness) $i \approx j , r_i < r_j \implies \forall k, l [r_k, r_l \in [r_i, r_j] \implies r_k \approx r_l ]$
%\end{itemize}
%




\begin{algorithm}
\caption{Oct 14 2021}
\label{algo1} 
\begin{algorithmic}[1]

\For{$i \in [n]$}
	\State{$\alpha \gets$ the current item to be decided}
	\If{$i < r$} \Comment Phase 1
		\If{$\exists \beta\in O_i \ldotp \alpha <_i \beta$}\Comment $\alpha$ is \emph{not} a maximum
			\State{Decline this item}
		\ElsIf{$\alpha \in E_i$}
			\State{Decline this item}
		\Else 
			\State{Examine this item}
		\EndIf
			
	\ElsIf{$i < n$}\Comment Phase 2
		\If{$\exists \beta\in O_i \ldotp \alpha <_i \beta$}
			\State{Decline this item}
		\Else				\If{$\forall \beta\in O_i \ldotp \beta \leq_i \alpha$} \Comment $\alpha$ is the absolute maximum
				\State{Accept this item}
			\ElsIf{$\alpha \in E_i$}
				\State{Decline this item}
			\Else
				\State{Examine this item}
			\EndIf
		\EndIf
	\Else
		\State{Accept this item}
	\EndIf

\EndFor

\end{algorithmic}
\end{algorithm}




\begin{algorithm}
\caption{Game Process, Relaxed}
\label{algo1} 
\begin{algorithmic}[1]


\State{$[n], r_i$ and $\approx$ relation are given according to definition}


\State{$O_1 = \{1\}$}
\State{$E_1 = \emptyset$}
\State{$\alpha_1 = 1$}
\State{$\Theta_1 = \emptyset$}



\For{$i \in [n]$}
	\State{$d_{i} \gets $ player's decision}
	\If{$d_i = \mathrm{Accept}$}
		\State Game terminates with reward $v(\alpha_i)$
			
	\ElsIf{$d_i = \mathrm{Decline}$}
		\State{$O_{i+1} \gets O_i\cup \{i+1\}$}
		\State{$E_{i+1} \gets E_i$}
		\State{$\alpha_{i+1} \gets i+1$}
		
	\ElsIf{$d_i = \mathrm{Examine}$}
		\State{$O_{i+1} \gets O_i$}
		\State{$E_{i+1} \gets E_i\cup \{i\}$}
		\State{$\alpha_{i+1} \gets \alpha$}
	\EndIf
	\State{$\Theta_{i+1} \gets 
		\{(\beta, \gamma) \in O_{i+1}^2: r_\beta \leq r_\gamma, \beta \not\approx \gamma\} \cup
		\{(\beta, \gamma) \in E_{i+1}^2: r_\beta \leq r_\gamma\}$}
	
	
\EndFor

\end{algorithmic}
\end{algorithm}

\newpage

\newcommand{\ALG}{\text{ALG}}
\newcommand{\OPT}{\text{OPT}}
\newcommand{\ind}{\textbf{1}}

We give the expected value of $\ALG$ on the relaxed model. Let $\alpha$ denote the selected item.

\begin{equation}
  \expect{\OPT}=m
\end{equation}

\begin{equation}
  \expect{\ALG}=\sum_{t=1}^m t\prob{v(\alpha)=t}
\end{equation}

\begin{align}
  \prob{v(\alpha)=t} &= \sum_{i=r}^n
  \prob{\alpha = i | v(i)=t}
  \prob{v(i)=t}\\&+
  \prob{\text{reach }n \cap \text{not accept }n} \prob{v(n)=t}\\&+
  \prob{\text{reach }n-1 \cap \text{examine and decline }n-1}\prob{v(n-1)=t}
\end{align}

The two added terms are cases where the items are exhausted. If $n$ is reached then it must be selected. If $n-1$ is examined then it must be selected also.

\begin{equation}
  \prob{v(i)=t}=1/m, \forall i\forall t
\end{equation}

\begin{align}
  \prob{\alpha = i | v(i)=t} &= \prob{\text{reach }i|\max_{j<r}v(j)\leq t}\prob{\max_{j<r}v(j)\leq t}\\&=
  \sum_{s=1}^t \prob{\text{reach }i|\max_{j<r}v(j)= s}\prob{\max_{j<r}v(j)= s}
\end{align}

\begin{equation}\label{eqn:exhaust1}
	\prob{\text{reach }n \cap \text{not accept }n}
	=\sum_{s=t+1}^m
		\prob{\text{reach }n|\max_{j<r}v(j)= s}\prob{\max_{j<r}v(j)= s}
\end{equation}

\begin{equation}\label{eqn:exhaust2}
	\prob{\text{reach }n-1 \cap \text{examine and decline }n-1}
	=\sum_{s=t+1}^m
		\ind_{s\leq t + \delta}
		\prob{\text{reach }n-1|\max_{j<r}v(j)= s}\prob{\max_{j<r}v(j)= s}
\end{equation}

\begin{align}
\prob{\max_{j<r}v(j)= s} 
	&= \prob{\max_{j<r}v(j)\leq s}-\prob{\max_{j<r}v(j)\leq s-1}\\
	&= (\frac{s}{m})^{r-1}-(\frac{s-1}{m})^{r-1}
\end{align}

\begin{align}
  \prob{\text{reach }i|\max_{j<r}v(j)= s}
  &= \sum_{(a_1,\dots,a_k):a_j\in\{1,2\},\sum a_j = i-r} \,\prod_{j=1}^k 
  \ind_{\delta<t}
  \left(\ind_{a_j = 1}\frac{s-\delta-1}{m}
  +\ind_{a_j = 2}\frac{\delta}{m}
  \right)
  +\ind_{\delta\geq t}\frac{t-1}{m}
  \\
  &= \sum_{x=0}^{\lfloor\frac{i-r}{2}\rfloor} 
  \left(
  		\ind_{\delta<t}\left(\frac{\delta}{m}\right)^x\left(\frac{s-\delta-1}{m}\right)^{i-r-2x}
  		+\ind_{\delta \geq x} \left(\frac{t-1}{m}\right)^x 0^{i-r-2x}
  \right)
  \frac{(i-r-x)!}{x!(i-r-2x)!}
\end{align}

Explanation: There are $i-r$ steps to go from $r$ to $i$. Each decline will go 1 step, and each examine+decline will go 2 steps. We calculate the probability to take each possible path and then sum them up. In the second equation, $x$ is the number of items examined going from $r$ to $i$.





\end{document} 