\documentclass{article}
\usepackage{algorithm}
\usepackage{algpseudocode}
\usepackage{pifont}
\usepackage{amsmath}
\usepackage{amsfonts}
\usepackage{mathtools}

\newcommand{\oP}{\operatorname{P}}

\newcommand{\expec}{\operatorname{E}}
\newcommand{\gt}{>}
\newcommand{\lt}{<}
\newcommand{\card}[1]{\lvert #1 \rvert}

\newcommand{\expect}{\operatorname{E}\expectarg}
\newcommand{\prob}{\operatorname{P}\expectarg}
\DeclarePairedDelimiterX{\expectarg}[1]{[}{]}{%
  \ifnum\currentgrouptype=16 \else\begingroup\fi
  \activatebar#1
  \ifnum\currentgrouptype=16 \else\endgroup\fi
}

\newcommand{\innermid}{\nonscript\;\delimsize\vert\nonscript\;}
\newcommand{\activatebar}{%
  \begingroup\lccode`\~=`\|
  \lowercase{\endgroup\let~}\innermid 
  \mathcode`|=\string"8000
}

\begin{document}

%$i \approx j$ means that $i$ and $j$ are in-distinguishable in the first observation.
%    
%    
%    \begin{itemize}
%  \item (reflexivity) $i \approx i$
%   \item(symmetry) $i \approx j \iff j \approx i$
%   \item(connectedness) $i \approx j , r_i < r_j \implies \forall k, l [r_k, r_l \in [r_i, r_j] \implies r_k \approx r_l ]$
%\end{itemize}
%




\begin{algorithm}
\caption{Oct 14 2021}
\label{algo1} 
\begin{algorithmic}[1]

\For{$i \in [n]$}
	\State{$\alpha \gets$ the current item to be decided}
	\If{$i < \lfloor \frac{n}{e}\rfloor$} \Comment Phase 1
		\If{$\exists \beta\in O_i \ldotp \alpha <_i \beta$}\Comment $\alpha$ is \emph{not} a maximum
			\State{Decline this item}
		\ElsIf{$\alpha \in E_i$}
			\State{Decline this item}
		\Else 
			\State{Examine this item}
		\EndIf
			
	\ElsIf{$i < n$}\Comment Phase 2
		\If{$\exists \beta\in O_i \ldotp \alpha <_i \beta$}
			\State{Decline this item}
		\Else				\If{$\forall \beta\in O_i \ldotp \beta \leq_i \alpha$} \Comment $\alpha$ is the absolute maximum
				\State{Accept this item}
			\ElsIf{$\alpha \in E_i$}
				\State{Decline this item}
			\Else
				\State{Examine this item}
			\EndIf
		\EndIf
	\Else
		\State{Accept this item}
	\EndIf

\EndFor

\end{algorithmic}
\end{algorithm}




\begin{algorithm}
\caption{Game Process}
\label{algo1} 
\begin{algorithmic}[1]


\State{$[n], r_i$ and $\approx$ relation are given according to definition}


\State{$O_1 = \{1\}$}
\State{$E_1 = \emptyset$}
\State{$\alpha_1 = 1$}
\State{$\Theta_1 = \emptyset$}



\For{$i \in [n]$}
	\State{$d_{i} \gets $ player's decision}
	\If{$d_i = \mathrm{Accept}$}
		\If{$\alpha_i = \mathrm{argmax}_j r_j$}
			\State Game terminates and player wins
		\Else
			\State Game terminates and player loses
			\EndIf
			
	\ElsIf{$d_i = \mathrm{Decline}$}
		\State{$O_{i+1} \gets O_i\cup \{i+1\}$}
		\State{$E_{i+1} \gets E_i$}
		\State{$\alpha_{i+1} \gets i+1$}
		
	\ElsIf{$d_i = \mathrm{Examine}$}
		\State{$O_{i+1} \gets O_i$}
		\State{$E_{i+1} \gets E_i\cup \{i\}$}
		\State{$\alpha_{i+1} \gets \alpha$}
	\EndIf
	\State{$\Theta_{i+1} \gets 
		\{(\beta, \gamma) \in O_{i+1}^2: r_\beta < r_\gamma, \beta \not\approx \gamma\} \cup
		\{(\beta, \gamma) \in E_{i+1}^2: r_\beta < r_\gamma\}$}
	
	
\EndFor

\end{algorithmic}
\end{algorithm}

\pagebreak

\newcommand{\ALG}{\text{ALG}}
\newcommand{\BestSoFar}{\pi_{i-1}}

We first observe that $O_i$ is a uniformly random sample from $[n]$. Therefore, the apparent rank of any upcoming item $\alpha_i$ is uniformly distributed in $[s_i]$, where $s_i = |O_i|$. We denote the best item observed so far by $\BestSoFar = \sup O_{i-1}$.

In particular, $\alpha_i >_r \BestSoFar$ with probability $1/s_i$. In phase 2, this is the necessary and sufficient condition for $\alpha_i$ to be accepted (A) by $\ALG$.

From now on we assume that $\alpha_i = i$. Define $N_j := \{k\in[n] | k \approx j\}$ be the neighbour of item $j$. Under the assumption that $\alpha \approx \beta \iff |r_\alpha - r_\beta|\leq \epsilon$, we have $|N_i|=2 \epsilon + 1$ whenever $r_i \in [\epsilon+1, n-\epsilon]$.

We put $i$ into one of four types. 
\begin{itemize}
\item It is a type-I item if $i = \sup O_i$ and $i \not \in N_{\sup O_{i-1}}$. A type-I item is examined and declined (ED) in phase I, and accepted (A) directly in phase II.

\item $i$ is a type-II item if $i = \sup O_i$ and $i \in N_{\sup O_{i-1}}$, which means that it's still the best ever and close enough to the previous best. A type-II item is ED in phase I, and examined and accepted (EA) in phase II. 

\item $i$ is a type-III item if $i \neq \sup O_i$ and $i \in N_{\sup O_{i-1}}$, which means that it's worse than the previous best but close enough to it. A type-III item is ED in both phase I and phase II. 

\item $i$ is a type-IV item if $i \neq \sup O_i$ and $i \not \in N_{\sup O_{i-1}}$, which means that it's much worse than the previous best and not even close to it. A type-IV item is declined directly (D) in both phase I and phase II. 
\end{itemize}

We already know $\prob{i \text{ is type-I or type-II}} = 1/s_i$.
Conditioned on $r_{\sup O_{i-1}} \in [\epsilon+1, n-\epsilon]$, $\prob{i \text{ is type-II or type III}}=\prob{i \in N_{\sup O_{i-1}}} = \frac{2\epsilon}{n-1}$, because $i$ is a random element picked from the remaining $(n-1)$ items. 
In case $r_{\sup O_{i-1}} < \epsilon$, $i$ cannot be type-I. In case $r_{\sup O_{i-1}} > n-\epsilon$, $i$ cannot be type-IV.

Furthermore, $\prob{i \text{ is type-II}} = \prob{i \text{ is type-III}}$ given $i \in [\epsilon, n-\epsilon]$. Combining all we know, we work out the probability for $i$ to fall in each type:
\begin{equation}
\begin{cases}

\text{Type I} & \max(1/s - p, 0)\\
\text{Type II} & \min(p, 1/s)\\
\text{Type III} & \min(p, n-1/s)\\
\text{Type IV} & \max(n-1/s-p, 0)

\end{cases}
\end{equation}


In case of D, $O_{i+1} = O_i + \{\alpha_i\}$ and the process continues for next $i$. ED forgoes one item un-observed. Hence in case of ED, $O_{i+2} = O_i + \{\alpha_i\}$ and the process continues for $i+2$.

Let $G(m)$ be the indicator of the event that $\ALG$ accepts any item before round $m$ (exclusive). Given a particular arrival order $\Theta_r$, the game state can be expressed by $(i,s)$ where $i$ is the current round and $s = |O_i|$. It can be solved by iterated equation:

\begin{gather*}
G(m) = \begin{cases}
g(1, 1) & 1\leq m\lt n\\
0 & m < 1\\
1 & m \geq n
\end{cases}\\
g(i,s) = \begin{cases}
(\mathbf{1}(A|\mathcal{H}_{is}) + 
\mathbf{1}(ED|\mathcal{H}_{is}))g(i+2, s+1) + 
\mathbf{1}(D|\mathcal{H}_{is})g(i+1,s+1) & i\leq\lfloor n/e \rfloor\\
\mathbf{1}(A|\mathcal{H}_{is}) + 
\mathbf{1}(ED|\mathcal{H}_{is})g(i+2, s+1) + 
\mathbf{1}(D|\mathcal{H}_{is})g(i+1,s+1) & \lfloor n/e \rfloor<i<m\\
0 & i \geq m
\end{cases}
\end{gather*}

$\mathcal{H}_{is}$ denotes all information available at state $(i,s)$, which is itself dependent on $\Theta_r$.
Take expectation over $\Theta_r$, and we obtain

\begin{equation*}
\expect{g(i,s)} = \begin{cases}
(1/s + 
p_{is})\expect{g(i+2, s+1)} + 
(1-1/s-p_{is})\expect{g(i+1,s+1)} & i\leq\lfloor n/e \rfloor\\

1/s + 
p_{is}\expect{g(i+2, s+1)} + 
(1-1/s-p_{is})\expect{g(i+1,s+1)} & \lfloor n/e \rfloor<i<m\\

0 & i \geq m
\end{cases}
\end{equation*}

where $p_{is} = \expec[\mathbf{1}(ED|\mathcal{H}_{is})]$, expectation taken over $\Theta_r$. We now give an upper bound to $p_{is}$. The upper bound will imply a lower bound to $\expec[\ALG]$.

\begin{align*}
p_{is} &= \expect[\big]{\expect{\mathbf{1}(ED(i)) |  \card{O_i}, \card{M_i}} | \card{O_i}}
\\&\leq \expect*{\frac{\card{M_{i-1}}}{\card{O_{i}} }| i,s}
\\&= \frac{\expect*{\card{M_{i-1}}|i,s}}{s}
\\&= \epsilon/n
\end{align*}

The first equality follows from definition and property of conditional expectation. The first inequality follows from (\ref{eq:ED given D bound}).  The last equality follows from the assumption that $\alpha \approx \beta \iff |r_\alpha - r_\beta|\leq \epsilon$.

Now we use the identity

\begin{equation*}
\expec[\ALG] = \expect*{\sum_{i=\lfloor n/e \rfloor}^n \left[1-G(i)\right] \left[G(i+1)-G(i)\right]F(i)}
\end{equation*}

$F$ is the conditional distribution $F(s | i, \text{no accept before }i)$ obtained in similar method as $G$.

\begin{gather*}
F(m) = \begin{cases}
f(1, 1) & 1\leq m\lt n\\
0 & m < 1\\
1 & m \geq n
\end{cases}\\
f(i,s) = \begin{cases} 
1+ \mathbf{1}(ED|\mathcal{H}_{is})f(i+2, s+1) + 
\mathbf{1}(D|\mathcal{H}_{is})f(i+1,s+1) & i\leq m \\
0 & i \gt m
\end{cases}
\end{gather*}



\end{document} 